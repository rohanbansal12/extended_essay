\subsection{Evenly-Split Batch Generator}
For improved stability of the training process and because of the skewed nature of the datasets, I chose to feed in batches that had an even number of positive and negative labels. To do this, I created a subclass of the PyTorch Sampler \parencite{NEURIPS2019_9015}.

\begin{minted}{python}
import torch
import numpy as np
import torch.nn as nn

# Create batches with even splits
# positive samples in first half and negative examples in second half
class BatchSamplerWithNegativeSamples(torch.utils.data.Sampler):
    def __init__(self, pos_sampler, neg_sampler, batch_size, items):
        self._pos_sampler = pos_sampler
        self._neg_sampler = neg_sampler
        self._items = items
        assert (
            batch_size % 2 == 0
        ), "Batch size must be divisible by two for negative samples."
        self._batch_size = batch_size

    def __iter__(self):
        batch, neg_batch = [], []
        neg_sampler = iter(self._neg_sampler)
        for pos_idx in self._pos_sampler:
            batch.append(pos_idx)
            neg_idx = pos_idx
            # keep sampling until we get a true negative sample
            while self._items[neg_idx] == self._items[pos_idx]:
                try:
                    neg_idx = next(neg_sampler)
                except StopIteration:
                    neg_sampler = iter(self._neg_sampler)
                    neg_idx = next(neg_sampler)
            neg_batch.append(neg_idx)
            if len(batch) == self._batch_size // 2:
                batch.extend(neg_batch)
                yield batch
                batch, neg_batch = [], []
        return

    def __len__(self):
        return len(self._pos_sampler) // self._batch_size

\end{minted}

\subsection{Batched Predictions}
These were the functions used to take an input batch of examples and generate predictions on them for both approaches. The device refers to a cpu or gpu depending on the hardware and target refers to the intended publication (which was the quality-writing publication, or id 0, for the entirety of this experiment).

\subsubsection{RankFromSets}
\begin{minted}{python}
@torch.no_grad()
def calculate_batched_predictions(batch, model, device, target=0):
    model.eval()
    (publications, articles, word_attributes, 
    attribute_offsets, real_labels) = batch
    publication_set = [target] * len(real_labels)
    publication_set = torch.tensor(publication_set, dtype=torch.long)
    publication_set = publication_set.to(device)
    articles = articles.to(device)
    word_attributes = word_attributes.to(device)
    attribute_offsets = attribute_offsets.to(device)
    logits = model(publication_set, articles, 
    			   word_attributes, attribute_offsets)
    final_logits = logits.cpu().numpy()
    return final_logits
\end{minted}

\subsubsection{BERT}
\begin{minted}{python}
@torch.no_grad()
def calculate_batched_predictions(batch, model, device, target=0):
    model.eval()
    (word_attributes, attention_masks, 
    word_subset_counts, real_labels) = batch
    word_attributes = word_attributes.to(device)
    attention_masks = attention_masks.to(device)
    logits = model(word_attributes, attention_mask=attention_masks)[0]
    final_logits = np.squeeze(logits.cpu().numpy())
    return final_logits
\end{minted}

\subsection{Recall Calculation}
This was the loop within a function that was used to generate recall statistics for the model on a new dataset (validation or test) with backpropagationa and gradient calculations disabled. 

\begin{minted}{python}
@torch.no_grad()
def calculate_recall(
    dataset, indices, recall_value=1000, 
    target_publication, version, writer, step,
):
    rev_indices = indices[::-1]
    correct_10 = 0
    correct_big = 0
    for i in range(recall_value):
        if dataset[rev_indices[i]]["model_publication"] == target_publication:
            if i < 10:
                correct_10 += 1
            correct_big += 1
    print(f"{version} Performance: Step - {step}")
    print(f"Top 10: {correct_10*10} %")
    print(
        f"Top {str(recall_value)}: {(correct_big*100)/recall_value} %"
    )
    print("--------------------")
    writer.add_scalar(f"{version}/Top-10", correct_10, step)
    writer.add_scalar(f"{version}/Top-{recall_value}", correct_big, step)
    return correct_big, correct_10
\end{minted}

\subsection{Generate Rankings List}
This was used to create a Pandas DataFrame that ranked results on a new dataset and outputted information on the top 1500 articles.

\begin{minted}{python}
def create_ranked_results_list(final_word_ids, sorted_preds, indices, data):
    df = pd.DataFrame(columns=["title", "url", 
    	"publication", "date", "prediction"])
    ranked_indices = indices[::-1]
    predictions = sorted_preds[::-1]
    for i in range(0, 1500):
        example = data[ranked_indices[i]]
        prediction = predictions[i]
        title = example["title"]
        url = example["link"]
        publication = example["publication"]
        date = example["date"]
        df.loc[i] = [title, url, publication, date, prediction]
    return df
\end{minted}