This essay is focused on comparing two modern machine learning approaches in their ability to classify and predict “quality” writing. This type of writing can be thought of as the material published in curated newsletters (such as The Browser) or quality-focused publications such as Longform.org. A concept such as quality-writing is inherently subjective due to its opinion-based nature, however, there are some notable aspects that seem to follow through what people typically regard as high-quality work. Pragmatic and unique descriptions, a strong topic and character development, and a logical structure are common keys in the analysis of good writing. Furthermore, scientists have found that content curators often use a very similar process when filtering articles, indicating that there is a programmable science to the art. This leads to the inference (and experiment) presented in this paper as a model with this ability can drastically reduce the workload of these editors while also providing a healthier newsfeed for indpendent consumers. There seems to be a general paucity of serious work in this space, partially because it may be quite difficult to train a machine to understand what we humans would consider fantastic literature and also because finding adequate data can be difficult and expensive. The two approaches considered are \gls{rfs}~\parencite{altosaar2020rankfromsets:} and \acrshort{bert}~\parencite{devlin2019bert:}, both novel machine-learning approaches. The primary method of comparison between the two models will come from their respective recall on the held-out evaluation set. The top 1000 predictions for each model will be considered, and the percentage of true positives will determine the model’s score. Other time and computing constraints will also be discussed throughout to provide enhanced context for all results. Hence, the question: \emph{How do transformer models with pre-contextualized word embeddings compare to set-based recommendation models, such as a dot-product model, in their ability to classify articles on the basis of the "quality" of their writing, as assessed by expert humans?}