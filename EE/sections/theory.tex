The concept of machine learning has become more popular for many tasks due to its ability to reduce human error and provide efficient methods for solving real-world problems. Prior to its advent, programming was generally a practice in which a developer gave instructions to a computer and the machine executed the commands it received. In contrast, machine learning teaches a computer to “think” or develop a model on its own that can then be utilized to tackle new situations. For example, classification-based systems, such as those being used in this paper, rely on large amounts of labeled data in the form of a \textbf{training dataset}. \textbf{Labels} refer to the category of each instance in the dataset, so they would be “quality” (0) or “regular” (1) here. The training set is fed to the model in \textbf{batches} (for computational speed and even label splits), where the model uses its \textbf{parameters} to generate predictions for the examples within the batch. These predictions are compared to the actual labels to calculate the \textbf{loss}, which measures the discrepancy between the predictions and intended results. Finally, an \textbf{optimizer} uses \textbf{back-propagation} (calculating the derivative of the final output with respect to each parameter) to update the parameters based on the loss. Machine learning also uses an evaluation and test dataset, which are used to monitor the progress of the model’s parameters. The model generates predictions on the evaluation set with a fixed frequency during training with back-propagation turned off. This allows the researcher to ensure that the model is not over-fitting on the training set, or becoming overly specific to the articles being trained on. Once evaluation performance begins to decrease, the model is used to generate predictions on a test set to get final results on completely new data. The model can then be used to generate new predictions on unlabelled data, such as for the purpose of a recommendation application system that fetches new articles daily. The specific subset of machine learning used in this paper is \gls{NLP} which pertains to text analysis.